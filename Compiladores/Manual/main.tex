% $Id: jfesample.tex,v 19:a118fd22993e 2013/05/24 04:57:55 stanton $
\documentclass[12pt]{article}

% DEFAULT PACKAGE SETUP

\usepackage{setspace,graphicx,epstopdf,amsmath,amsfonts,amssymb,amsthm,versionPO}
\usepackage{marginnote,datetime,enumitem,subfigure,rotating,fancyvrb}
\usepackage{hyperref,float}
\usepackage[longnamesfirst]{natbib}
\usdate

% These next lines allow including or excluding different versions of text

\excludeversion{notes}		% Include notes?
\includeversion{links}          % Turn hyperlinks on?

% Turn off hyperlinking if links is excluded
\iflinks{}{\hypersetup{draft=true}}

% Notes options
\ifnotes{%
\usepackage[margin=1in,paperwidth=10in,right=2.5in]{geometry}%
\usepackage[textwidth=1.4in,shadow,colorinlistoftodos]{todonotes}%
}{%
\usepackage[margin=1in]{geometry}%
\usepackage[disable]{todonotes}%
}

% Allow todonotes inside footnotes without blowing up LaTeX
% Next command works but now notes can overlap. Instead, we'll define 
% a special footnote note command that performs this redefinition.
%\renewcommand{\marginpar}{\marginnote}%

% Save original definition of \marginpar
\let\oldmarginpar\marginpar

% Workaround for todonotes problem with natbib (To Do list title comes out wrong)
\makeatletter\let\chapter\@undefined\makeatother % Undefine \chapter for todonotes

% Define note commands
\newcommand{\smalltodo}[2][] {\todo[caption={#2}, size=\scriptsize, fancyline, #1] {\begin{spacing}{.5}#2\end{spacing}}}
\newcommand{\rhs}[2][]{\smalltodo[color=green!30,#1]{{\bf RS:} #2}}
\newcommand{\rhsnolist}[2][]{\smalltodo[nolist,color=green!30,#1]{{\bf RS:} #2}}
\newcommand{\rhsfn}[2][]{%  To be used in footnotes (and in floats)
\renewcommand{\marginpar}{\marginnote}%
\smalltodo[color=green!30,#1]{{\bf RS:} #2}%
\renewcommand{\marginpar}{\oldmarginpar}}
%\newcommand{\textnote}[1]{\ifnotes{{\noindent\color{red}#1}}{}}
\newcommand{\textnote}[1]{\ifnotes{{\colorbox{yellow}{{\color{red}#1}}}}{}}

% Command to start a new page, starting on odd-numbered page if twoside option 
% is selected above
\newcommand{\clearRHS}{\clearpage\thispagestyle{empty}\cleardoublepage\thispagestyle{plain}}

% Number paragraphs and subparagraphs and include them in TOC
\setcounter{tocdepth}{2}

% JFE-specific includes:

\usepackage{indentfirst} % Indent first sentence of a new section.
\usepackage{jfe}          % JFE-specific formatting of sections, etc.

\newtheorem{theorem}{Theorem}[section]
\newtheorem{assumption}{Assumption}[section]
\newtheorem{proposition}{Proposition}
\newtheorem{conjecture}{Conjecture}
\newtheorem{lemma}{Lemma}[section]
\newtheorem{corollary}{Corollary}
\newtheorem{condition}{Condition}

\begin{document}

\setlist{noitemsep}  % Reduce space between list items (itemize, enumerate, etc.)
%\onehalfspacing      % Use 1.5 spacing
% Use endnotes instead of footnotes - redefine \footnote command

\title{Tokuwaga\\Compilador de la M\'aquina Sencilla}

\author{Dise\~no de Sistemas con FPGA\\
  Departamento de Computaci\'on\\
  Facultad de Ciencias Exactas y Naturales\\
  Universidad de Buenos Aires}

\date{Primer Cuatrimestre, 2016}          % No date for final submission

\renewcommand*\contentsname{\'Indice}

% Create title page with no page number

\singlespacing

\maketitle

\vspace{1in}
\begin{abstract}
\noindent Implementaci\'on de un compilador para la m\'aquina sencilla, a\~nadir funciones al repertorio de la m\'aquina mediante pseudo instrucciones, optimizaci\'on del uso de memoria.
\end{abstract}

\medskip
\noindent \textit{Palabras claves}: compilador, pseudo-instrucciones, lenguaje ensamblador, lenguaje m\'aquina, parser, desensamblador.

\vspace{2in}
\noindent 
Ishikame, Emiliano \ \ \ - \ \ xxx/xx \ \ - emilianoishikame@yahoo.com.ar\\
Torrico, Myrko - \ \ xxx/xx \ \ -mirko.torrico@gmail.com\\

\medskip

\thispagestyle{empty}

\clearpage

\tableofcontents

\clearpage

\setcounter{page}{1}

\section{Introducci\'on}

Nuestro ensamblador, Tokuwaga, se encarga de recibir un archivo.asm y generar su transformación a lenguaje de máquina. El archivo recibido pasa por  el siguiente proceso: 

Limpia el c\'odigo: se deshace de comentarios y reconoce variables, constantes y etiquetas para poder direccionarlas despu\'es, y hace el reemplazo de las pseudo-instrucciones. 
Optimiza Memoria. 
Parsea el c\'odigo.

En el inicio, Tokuwaga soportaba s\'olo las instrucciones de nuestra M\'aquina sencilla (“MOV”,”ADD”,”CMP”, etc). Luego para facilitar la programaci\'on en assembler, nos vimos obligados a aceptar pseudo-instrucciones como: “JUMP”, “SUB”, “LEA”, entre otros.

Finalmente Tokuwaga  permite el uso de constantes, usando como prefijo @; variables  y etiquetas de la pinta etiqueta:. Para el testeo tambi\'en se desarroll\'o un desensamblador, el cual transforma el lenguaje m\'aquina en c\'odigo ensamblador.

\section{Operaciones}
Lenguaje del miniCompilador
Instrucciones v\'alidas de ensamblador

\subsection{DEFINICI\'ON DE TIPOS}

\begin{tabular}{| c | l |}
\hline 
P & $DEVICE\_DIR|REG\_SEL$  \\ \hline
D & Direccion Destino  \\  \hline
S & Direccion Fuente \\ \hline
A & ADDR\\  \hline
@12 & constante (DECIMAL)\\\hline
\end{tabular}

\subsection{OPERACIONES ARIM\'ETICO/L\'OGICAS}     

\begin{tabular}{| c | c |p{5.5cm}|}
\hline 
OPCODE & Funci\'on & Descripci\'on \\ \hline
ADD S,D & [D]=[S]+[D],Z=([S]+[D])==0 & Suma S+D y lo almacena en D, Si S+D=0 entonces flag Z es 1 si no flag Z=0.\\ \hline
CMP S,D &  Z=([S]-[D])==0 & Si S-D=0 , entonces el flag Z=1 si no flag Z = 0.\\  \hline
DEC D & [D]=[D]-1 & Decrementa la variable D en 1. \\ \hline
INC D & [D]=[D]-1 & Incrementa la variable D en 1 (PSEUDO).\\  \hline
SUB D,S & [D]=[D]-[S] & Resta S a D y lo guarda en D.\\ \hline
SHIFTL A,S & [A]=A$>>$S & Aplica un shift l\'ogico a izquierda sobre A tantos bits como indica S.\\ \hline
SHIFTR A,S & [A]=A$<<$s & Aplica un shift l\'ogico a derecha sobre A tantos bits como indica S.\\ \hline
\end{tabular}

\subsection{MOVIMIENTO DE DATOS}

\begin{tabular}{| c | c |p{8.5cm}|}
\hline 
OPCODE & Funci\'on & Descripci\'on \\ \hline
MOV S,D & [D]=[S]  & Mueve lo que esta en la posici\'on de memoria S a la posici\'on D.\\ \hline
MOV S,[D] &  [D]=[[S]]  & Mueve lo que esta en la posici\'on de memoria apuntada por S a la posici\'on de memoria D.\\  \hline
MOV S,[D] & [[D]]=[S]  & Mueve lo que esta en la posici\'on de memoria S a la posici\'on apuntada por D. \\ \hline
MOV [S],[D] & [[D]]=[[S]]  & Mueve lo que esta en la posici\'on apuntada por S a la posici\'on apuntada por D. \\ \hline
LEA "TAG",D & [D]="TAG" & Carga la direcci\'on en que se encuentra la etiqueta,variable o constante "TAG" en la posici\'on D.\\  \hline
\end{tabular}


\subsection{OPERACI\'ON DE ENTRADA/SALIDA}
\begin{tabular}{| c | c |p{9.5cm}|}
\hline 
OPCODE & Funci\'on & Descripci\'on \\ \hline
IN P,D & [D]=[S]  & Mueve a la posici\'on de memoria D lo que esta en el puerto P.\\ \hline
OUT P,S &  [D]=[[S]]  & Mueve al puerto P lo que esta en la memoria en la posici\'on S.\\  \hline
\end{tabular}


\subsection{OPERACIONES DE SALTO}

\begin{tabular}{| c | c |p{5cm}|}
\hline 
OPCODE & Funci\'on & Descripci\'on \\ \hline
BEQ "TAG" & Z==1 $\rightarrow$ PC="TAG" & salta si el flag Z esta activo a la posici\'on de la etiqueta o posici\'on "TAG".\\ \hline
CALL "TAG" &  [[reg]=PC, PC="TAG", reg=reg-1   & Llama a la funci\'on "TAG".\\  \hline
RET & reg+1, PC=[reg]   & Retorna de un call. \\ \hline
JMP "TAG" & PC="TAG" & Salto a la posici\'on de la etiqueta o posici\'on "TAG.\\  \hline
\end{tabular}

\subsection{DEFINICI\'ON DE DATOS}
\begin{tabular}{| c | p{11.5cm}|}
\hline 
OPCODE & Descripci\'on  \\ \hline
DW xxx &  Define un word de 16 bits en esa posici\'on de memoria (el n\'umero ingresado debe ser decimal).\\ \hline
LABEL: &  Define una etiqueta.\\  \hline
\end{tabular}


Soporta uso de variables y constantes

\section{Instrucciones en orden alfab\'etico}

\subsection{Instrucci\'on ADD}
El Destino de la funci\'on ADD no puede ser una constante.\\
\begin{tabular}{| c | p{11.5cm}|}
\hline 
OPCODE & Descripci\'on  \\ \hline
ADD a,b & Suma la variable a con b y lo guarda en b.\\ \hline
ADD a,1 & Suma la posici\'on 1 con a y lo guarda en la posici\'on 1.\\ \hline
ADD 1,a & Suma la variable a con 1 y lo guarda en la variable a.\\ \hline
ADD @1,a & Suma la variable a con la constante y lo guarda en la variable a.\\ \hline
\end{tabular}

\subsection{Instrucci\'on BEQ}

\begin{tabular}{| c | p{11.5cm}|}
\hline 
OPCODE & Descripci\'on  \\ \hline
BEQ 1     & si la comparaci\'on FZ=1 salta a la posici\'on 1.\\ \hline
BEQ LABEL & si la comparaci\'on FZ=1 salta a la etiqueta LABEL.\\ \hline
\end{tabular}

\subsection{Instrucci\'on CALL }
\begin{tabular}{| c | p{11.5cm}|}
\hline 
OPCODE & Descripci\'on  \\ \hline
CALL LABEL & Salta a la etiqueta LABEL.\\ \hline
CALL 100 & Salta a la posici\'on 100  (valor absoluto).\\ \hline
\end{tabular}

\subsection{Instrucci\'on CMP }

\begin{tabular}{| c | p{11.5cm}|}
\hline 
OPCODE & Descripci\'on  \\ \hline
CMP a,b & Compara las variables a y b y guarda el resultado en el flag Z.\\ \hline
CMP a,1 & Compara a y con la posicion de memoria 1 y guarda el resultado en el flag Z.\\ \hline
CMP 1,2 & Compara la posici\'on de memoria 1 y 2 y guarda el resultado en el flag Z.\\ \hline
CMP @1,a & Compara una constante con la variable a y guarda el resultado en el flag Z.\\ \hline
\end{tabular}

\subsection{Instrucci\'on DEC }
El Destino de la funci\'on DEC no puede ser una constante.\\
\begin{tabular}{| c | p{11.5cm}|}
\hline 
OPCODE & Descripci\'on  \\ \hline
DEC a & incrementa la variable a en 1.\\ \hline
DEC 0 & incrementa lo que hay en la posici\'on 0 en 1.\\ \hline
\end{tabular}

\subsection{Instrucci\'on DW }
Define una constante.

\subsection{Instrucci\'on INC }
El Destino de la funci\'on INC no puede ser una constante.\\
\begin{tabular}{| c | p{11.5cm}|}
\hline 
OPCODE & Descripci\'on  \\ \hline
INC a & incrementa la variable a en 1.\\ \hline
INC 0 & incrementa lo que hay en la posici\'on 0 en 1.\\ \hline
\end{tabular}

\subsection{Instrucci\'on IN/OUT }
El Destino de la funci\'on IN no puede ser una constante.

\begin{tabular}{| c | p{11.5cm}|}
\hline 
OPCODE & Descripci\'on  \\ \hline
IN puerto,10 & mueve el puerto a la posici\'on de memoria 10.\\ \hline
IN puerto,D  & mueve el puerto a la variable S.\\ \hline
OUT puerto,10 & mueve la posici\'on de memoria 10 al puerto.\\ \hline
OUT puerto,S  & mueve la variable S al puerto.\\ \hline
OUT puerto,@10  & mueve una constante al puerto.\\ \hline
\end{tabular}

El puerto tiene direcciones validas de 0 a 31. Para los dispositivos predefinidos hay etiquetas para facilitar la identificaci\'on.
\begin{itemize}
    \item LED
    \item PUERTO\_0\_SHIFTER
    \item PUERTO\_1\_SHIFTER
    \item PUERTO\_2\_SHIFTER
    \item PUERTO\_3\_SHIFTER
    \item UART\_TX
    \item UART\_RX
    \item UART\_FULL
    \item UART\_EMPTY
    \item SSEG
    \item BTNS
    \item SWT
\end{itemize}

\subsection{Instrucci\'on JMP }
El Destino de la funci\'on IN no puede ser una constante.\\
\begin{tabular}{| c | p{11.5cm}|}
\hline 
OPCODE & Descripci\'on  \\ \hline
JMP label & salta a la etiqueta label.\\ \hline
JMP 10  & mueve el puerto a la variable S.\\ \hline
\end{tabular}

\subsection{Instrucci\'on LEA}
Permite obtener la direcci\'on donde el ensamblador almacena una variable.\\
\begin{tabular}{| c | p{11.5cm}|}
\hline 
OPCODE & Descripci\'on  \\ \hline
LEA LABEL,A & Carga la posici\'on de la etiqueta LABEL en A.\\ \hline
LEA b,A     & Carga la posici\'on de la variable b en A.\\ \hline
LEA @100,A  & Carga la posici\'on de la constante 100 a A.\\ \hline
\end{tabular}

\subsection{Instrucci\'on MOV}
El Destino de la funci\'on MOV no puede ser una constante.\\
Si el par\'ametro pasado por referencia esta fuera de los l\'imites de memoria, la funci\'on no garantiza su ejecuci\'on y/o integridad de la memoria.\\
\begin{tabular}{| c | p{11.5cm}|}
\hline 
OPCODE & Descripci\'on  \\ \hline
MOV S,D & Mueve de S a D..\\ \hline
MOV S,1  & Mueve lo que esta en la "variable" S a la posici\'on 1.\\ \hline
MOV 1,D  & Mueve lo que esta en la posicion 1 a la variable D.\\ \hline
MOV @10,S  & Mueve una constante a la variable S.\\ \hline
MOV @10,1  & Mueve una constante a la posici\'on de memoria 1.\\ \hline
MOV [S],1  & Mueve lo que esta en la posici\'on referenciada por S a la posici\'on 1.\\ \hline
MOV [1],D  & Mueve lo que esta en la posici\'on referenciada por la posici\'on 1 a la variable D.\\ \hline
MOV @10,[D] & Mueve una constante a la posici\'on referenciada por D.\\ \hline
MOV @10,[1] &Mueve una constante a la posici\'on referenciada por la posici\'on 1.\\ \hline
MOV @10,[@14] & Mueve una constante a la posici\'on referenciada por la constante (equivale a MOV @10,14).\\ \hline
MOV S,[@14]  & Mueve lo que esta en la "variable" S a la posici\'on referenciada por la constante (equivale a MOV @10,14).\\ \hline
\end{tabular}

\subsection{Instrucci\'on RET }
Retorna de la ejecucion de un call.

\subsection{Instrucci\'on SHIFT }
El Destino de la funci\'on SHIFT no puede ser una constante.\\
\begin{tabular}{| c | p{11.5cm}|}
\hline 
OPCODE & Descripci\'on  \\ \hline
SHIFTR X,10 & Desplaza X l\'ogicamente a derecha tantos bits como indica la posici\'on de memoria 10.\\ \hline
SHIFTR X,S  & Desplaza X l\'ogicamente a derecha tantos bits como indica la variable S.\\ \hline
SHIFTR X,@10 & Desplaza X l\'ogicamente a derecha tantos bits como indica la constante.\\ \hline
\hline 
SHIFTL X,10 & Desplaza X l\'ogicamente a izquierda tantos bits como indica la posici\'on de memoria 10.\\ \hline
SHIFTL X,S  & Desplaza X l\'ogicamente a izquierda tantos bits como indica la variable S.\\ \hline
SHIFTL X,@10 & Desplaza X l\'ogicamente a izquierda tantos bits como indica la constante.\\ \hline \hline
SHIFTR 25,10 & Desplaza lo que hay en la posici\'on 25 l\'ogicamente a derecha tantos bits como indica la posici\'on de memoria 10.\\ \hline
SHIFTR 25,S  & Desplaza lo que hay en la posici\'on 25 l\'ogicamente a derecha tantos bits como indica la variable S.\\ \hline
SHIFTR 25,@10 & Desplaza lo que hay en la posici\'on 25 l\'ogicamente a derecha tantos bits como indica la constante.\\ \hline
\hline 
SHIFTL 25,10 & Desplaza lo que hay en la posici\'on 25 l\'ogicamente a izquierda tantos bits como indica la posici\'on de memoria 10.\\ \hline
SHIFTL 25,S  & Desplaza lo que hay en la posici\'on 25 l\'ogicamente a izquierda tantos bits como indica la variable S.\\ \hline
SHIFTL 25,@10 & Desplaza lo que hay en la posici\'on 25 l\'ogicamente a izquierda tantos bits como indica la constante.\\ \hline
\end{tabular}
\\Los valores insertados son siempre 0, el m\'aximo valor de desplazamiento es 15.
Si se excede este l\'imite la funci\'on puede no devolver un valor o provocar alteraciones en la memoria.

\subsection{Instrucci\'on SUB}
El Destino de la funci\'on SUB no puede ser una constante.\\
\begin{tabular}{| c | p{11.5cm}|}
\hline 
OPCODE & Descripci\'on  \\ \hline
SUB X,Y & Resta la variable Y a la variable X y lo guarda en X.\\ \hline
SUB X,1 & Resta el valor que hay en la posici\'on 1 a la variable X y lo guarda en X.\\ \hline
SUB X,@1 & Resta la constante 1 a X y lo guarda en X.\\ \hline
SUB 1,X & Resta la variable X al valor que hay en la posici\'on 1 y lo guarda en la posici\'on 1.\\ \hline
SUB 1,@1 & Resta la constante 1 al valor que hay en la posici\'on 1 y lo guarda en la posici\'on 1.\\ \hline
SUB 1,12 & Resta el valor que hay en la posici\'on 12 al valor que hay en la posici\'on 1 y lo guarda en la posici\'on 1.\\ \hline
\end{tabular}


\section{Pseudo-Instrucci\'on}
\subsection{Introducci\'on}
Dado que la m\'aquina tiene pocas instrucciones disponibles, para facilitar el trabajo al programador, se decide proveer de estas instrucciones. Esto no significa que sea necesariamente mejor que la que pueda implementar. Brinda este servicio de forma general, por lo que puede no ser \'optimo.
\subsection{Desarrollo}
\subsubsection{INC}
La funcion INC se implementa mediante la creaci\'on de la constante 1 y utilizando la instrucci\'on ADD. De modo que INC x se transforma en ADD @1,X
\begin{verbatim}
INC X   =       ADD c,X
            c:  DW 1
\end{verbatim} 
\subsubsection{DEC}
La funcion INC se implementa mediante la creaci\'on de la constante 65535 y utilizando la instrucci\'on ADD. De modo que INC x se transforma en ADD @1,X
\begin{verbatim}
DEC X   =       ADD c,X
            c:  DW 65535
\end{verbatim} 
\subsubsection{SUB}
La funcion SUB se implementa mediante una resta sucesiva hasta que el segundo operando es cero, luego se mueve el resultado al  segundo operando.
\begin{verbatim}
    SUB A,B =           MOV B,Var 
                loop:   CMP @0,Var 
                        BEQ fin 
                        ADD @65535,Var
                        ADD @65535,A 
                        CMP 0,0
                        BEQ loop
                fin:
\end{verbatim}
\subsubsection{LEA}
La funcion LEA se implementa mediante la creaci\'on de la constante c, la cual es la posici\'on donde se encuentra X y utilizando la instrucci\'on MOV. De modo que LEA x,0 se transforma en MOV c,X
\begin{verbatim}
LEA c,X     =   MOV a,X
                a:posicion de memoria de c asignada por el compilador
\end{verbatim}  
\subsubsection{MOV}
La funcion MOV con indirecci\'on se implementa mediante un algoritmo que modifique la memoria para que la siguiente instrucci\'on que se ejecute sea un MOV con indireccion.\\
 MOV [A],B
\begin{verbatim}
            MOV A,X 
            MOV @0,Contador
    loop:   CMP CONTADOR_LOOP,7
            BEQ FIN
            ADD X,X
            ADD @65535,Contador
            CMP 0,0
            BEQ loop
    FIN:    ADD @32768,X
            LEA B,var
            ADD var,X       
    X:      MOV A,0
\end{verbatim}    
 MOV [A],[B]
\begin{verbatim}
            MOV A,X 
            MOV @0,Contador Loop
    loop:   CMP CONTADOR_LOOP,7
            BEQ FIN
            ADD X,X
            INC contador_loop
            CMP 0,0
            BEQ loop
    FIN     ADD @32768,X
            ADD B,X
    X:      MOV A,0
\end{verbatim} 
 MOV A,[B]
\begin{verbatim}
            LEA A,X 
            MOV @0,Contador
    loop:   CMP CONTADOR_LOOP,7
            BEQ FIN
            ADD X,X
            ADD @65535,Contador
            CMP 0,0
            BEQ loop
    FIN:    ADD @32768,X
            LEA B,var
            ADD var,X       
    X:      MOV A,0
\end{verbatim}

Aprovechado los modulos de E/S, se puede simplificar la instrucci\'on

MOV [A],B
\begin{verbatim}
            OUT PUERTO_0_SHIFTER,@7
            OUT PUERTO_1_SHIFTER,A
            IN  PUERTO_2_SHIFTER,X
            ADD @32768,X
            LEA B,var
            ADD var,X       
    X:      MOV A,0
\end{verbatim}    

 MOV [A],[B]
\begin{verbatim}
            OUT PUERTO_0_SHIFTER,@7
            OUT PUERTO_1_SHIFTER,A
            IN  PUERTO_2_SHIFTER,X
            ADD @32768,X
            ADD B,X
    X:      MOV A,0
\end{verbatim} 
 MOV A,[B]
\begin{verbatim}
            LEA A,X 
            OUT PUERTO_0_SHIFTER,@7
            OUT PUERTO_1_SHIFTER,X
            IN  PUERTO_2_SHIFTER,X
            ADD @32768,X
            ADD B,X
    X:      MOV A,0

\subsubsection{SHIFT}
La funcion SHIFT era parcialmente soportada hasta que fue a\~nadido el m\'odulo de 
entrada/salida que permite dicha operaci\'on.

SHIFTR X,@const
\begin{verbatim}
    OUT PUERTO_0_SHIFTER,@const
    OUT PUERTO_1_SHIFTER,X
    IN  PUERTO_3_SHIFTER,X
\end{verbatim}

SHIFTR X,Y
\begin{verbatim}
    OUT PUERTO_0_SHIFTER,Y
    OUT PUERTO_1_SHIFTER,X
    IN  PUERTO_3_SHIFTER,X
\end{verbatim}

SHIFTL X,@const
\begin{verbatim}
    OUT PUERTO_0_SHIFTER,@const
    OUT PUERTO_1_SHIFTER,X
    IN  PUERTO_2_SHIFTER,X
\end{verbatim}

SHIFTL X,Y
\begin{verbatim}
    OUT PUERTO_0_SHIFTER,Y
    OUT PUERTO_1_SHIFTER,X
    IN  PUERTO_2_SHIFTER,X
\end{verbatim}

\section{Compilaci\'on}
El m\'etodo empleado para compilar el programa es el escaneo en 2 pasadas. En la
primer lectura, se reconocen las etiquetas, variables y constantes y se les asigna
un token. Luego, corremos un algoritmo que busca minimizar la cantidad de memoria
utilizada por las variables y constantes. Finalmente, en la segunda lectura se
procede a reemplazar las l\'ineas por c\'odigo m\'aquina y para resolver las
posiciones desconocidas, utiliza el diccionario creado en el paso anterior.

\section{Optimizaci\'on del uso de memoria}
\subsection{Motivaci\'on}
Dado que la m\'aquina tiene memoria finita y, en este caso, muy acotada,
 se debe aprovechar al m\'aximo este recurso.

\subsection{Desarrollo}
El problema de memoria se intent\'a mejorar desde dos puntos de vista: instrucciones
y variables. Por el lado de las instrucciones, se busca que las pseudo instrucciones
ocupen poca memoria y tengan un rendimiento aceptable. Por el otro lado, se intenta
aprovechar la memoria organizando las constantes y variables, de forma que esta sea
un conjunto minimal y de menor tama\~no posible.

\section{Testing}
Para testear el c\'odigo generado por el compilador, se implement\'o un reensamblador,
el cual genera instrucciones (si las existen) a partir del c\'odigo binario. Dado que
las pseudo instrucciones no pueden ser reensambladas y las variables y constantes no
son visibles en el c\'odigo original, el c\'odigo obtenido es, en general, de mayor
tama\~no. Adem\'as, como algunas pseudo-instrucciones crean la instrucci\'on en el momento de ejecuci\'on, se implement\'o un simulador de la m\'aquina sencilla para poder realizar el seguimiento sobre dichas instrucciones.

\section{Mejoras posibles en el futuro}

Se puede mejorar la velocidad de compilaci\'on mejorando el modo de compilaci\'on. Por otro lado, se puede mejorar la distribuci\'on de las variables para consumir menos memoria.  Por motivos de r\'apida implementaci\'on el compilador, el reesamblador y el simulador fueron creados independientemente, creemos que integrar las tres herramientas en un solo entorno permitira crear c\'odigo para la m\'aquina sencilla de manera simple. Adem\'as, a causa de que el simulador no ofrece la posibilidad que se realicen comandos de entrada, se limita las pruebas de programas a procesamiento est\'atico. Ser\'ia id\'oneo que en el futuro admita esa funcionalidad.

\section{NOTAS IMPORTANTES	}
\begin{itemize}
    \item Es responsabilidad del programador que las instrucciones de SHIFT,MOV
si se ejecutan pasando un par\'ametro por referencia, esta este dentro 
del rango v\'alido.
    \item Las pseudo instrucciones no necesariamente ocupan 1 palabra en memoria.
    \item Cuando las pseudo-instrucciones son utilizadas, tener cuidado con las referencias absolutas, es recomendable utilizar etiquetas.
\end{itemize}

\fbox{\parbox[b]{\linewidth}{\textbf{\underline{\emph{Importante:}}} UNA ETIQUETA
 TIENE PRIORIDAD SOBRE UNA VARIABLE, SI EL NOMBRE DE UNA VARIABLE ES EL MISMO QUE UNA
 ETIQUETA, CUANDO SE REFERENCIE APUNTAR\'A LA ETIQUETA.}}

\section{BUGS}
Lista de Known Issues:
\begin{itemize}
    \item No soporta dos instrucciones v\'alidas en la misma l\'inea.
    \item Hay una variable de ensamblador que est\'a reservada.
    \item Si bien hace un chequeo general de los operandos, puede que acepte algo
            no soportado.
\end{itemize}

\end{document}
